\documentclass[12pt,a4paper]{article}
\usepackage[utf8]{inputenc}
\usepackage{amsmath}
\usepackage{amsfonts}
\usepackage{amssymb}
\usepackage{graphicx}
\usepackage{xspace}
\usepackage[left=2cm,right=2cm,top=2cm,bottom=2cm]{geometry}
\author{Jost Berthold}

\title{Control stacks (Chapter 28 PFPL)\\
       {\small -- in a more accessible notation --}
}

\begin{document}
%\maketitle

\newcommand{\PCF}{{\it PCF}\xspace}

\section*{\PCF (refresh)}

\begin{align*}
 Var, \{z\} & \subset \PCF\\
s(e) & \in \PCF \mbox{ for } e \in \PCF\\
\mbox{ifz}\,(e_1, x.e_2) &\in \PCF \mbox{ for } e_1, e_2 \in \PCF,\, x\in Var\\
\mbox{lam}\,(x.e) &\in \PCF \mbox{ for } e \in \PCF,\, x\in Var\\
\mbox{ap}\,(e_1,e_2) &\in \PCF \mbox{ for } e_1, e_2 \in \PCF\\
\mbox{fix}\,(x.e) &\in \PCF \mbox{ for } e \in \PCF,\, x\in Var\\
\end{align*}

Expressions are analysed for consistent types, and evaluation (to what is 
judged a \texttt{val}, \\textit{z, s(e), lam}) proceeds based on evaluation
rules. While every rule does have just one premise that would require
a sub-evaluation, this sub-evaluation is assumed to magically be finished and
the outcome known, which makes it unfavourable for implementation. 

\section*{K machine}

The K machine evaluates PCF programs using a control stack instead of
context provided by premises in rules. Thereby, a practical implementation
can be derived directly from its definition.

\subsection*{Domains and states}

\begin{itemize}
\item Expressions: $e \in \PCF$
\item Stack frames: $$ F = \{ s(-)\} \cup \{ ap(-, e) | e \in \PCF \}
                     \cup \{ {\it ifz}(e_1, x.e_2)(-) | e_1, e_2 \in \PCF; x \in Var \} $$
\item Stack: $St = F^*$ a list of stack frames, possibly empty ($\varepsilon$)

    My stack grows to the left (and uses (:) cons), whereas Harper's grows to the right.

\item Machine state: $ S = St \times \{ \triangleleft, \triangleright\} \times \PCF $ 

    $\triangleleft$ indicates use of the topmost stack frame,
    $\triangleright$ indicates \PCF evaluation.
\end{itemize}


\subsection*{Transitions}
The K machine is based on an evaluation relation on machine states, $\mapsto \in S \times S$:

\newcommand{\eval}[2]{#1 \, \triangleright \, #2}
\newcommand{\stack}[2]{#1 \, \triangleleft \, #2}

\begin{itemize}
\item Numbers:
\[
\begin{array}{rcl}
    \eval{k}{z} &\mapsto& \stack{k}{z}\\
    \eval{k}{s(e)} &\mapsto& \eval{s(-):k}{e}\\
    \stack{s(-):k}{e} &\mapsto& \stack{k}{s(e)}\\
\end{array}
\]
\item Branching:
\[
\begin{array}{rcl}
    \eval{k}{{\it ifz}(e_1, x.e_2)(e)} &\mapsto& \eval{{\it ifz}(e_1, x.e_2)(-):k}{e}\\
    \eval{{\it ifz}(e_1, x.e_2)(-): k}{z} &\mapsto& \eval{k}{e_1}\\
    \eval{{\it ifz}(e_1, x.e_2)(-): k}{s(e)} &\mapsto& \eval{k}{[e/x]\,e_2}\\
\end{array}
\]
\item Functions:
\[
\begin{array}{rcl}
    \eval{k}{lam(x.e)} &\mapsto& \stack{k}{lam(x.e)}\\
    \eval{k}{ap(e_1, e_2)} &\mapsto& \eval{ap(-, e_2): k}{e_1}\\
    \eval{ap(-, e_2):k}{lam(x.e)} &\mapsto& \eval{k}{[e_2/x]\,e}\\
    \eval{k}{fix(x.e)} &\mapsto& \eval{k}{[fix(x.e)/x]\,e}\\
\end{array}
\]
\end{itemize}

Somewhat strangely, functions are applied to unevaluated expression
arguments (call by name). Other than that, no surprises.

\paragraph*{Start and End}
The machine evaluates an $e \in \PCF$ by starting in state
$S_{start} = \eval{\varepsilon}{e}$ and performing
evaluations from $\mapsto$ until end state
$ \stack{\varepsilon}{e}$ with a \underline{value} $e$.

\subsection*{Safety}
\textbf{Safety} is argued for by a type-like property of stack frames.

Stack frames "expect" a type (obvious definition), and usually "apply" themselves
to transform that expected type to another.
Stack frames are stacked up with matching types according to this idea:

\begin{quote}
If stack $k$ expects $\tau'$ and frame $f$ expects $\tau$ and yields $\tau'$,
then stack $f:k$ is well-formed and expects $tau$. The empty stack can expect any type as required.

\end{quote}

The stack frame type must match the type of the expression under evaluation
at all times.

The proof of safety relates well-formed stacks and matching expressions to 
the evaluation relation: 
Shows that the K machine won't get stuck in K machine states without successor
(i.e. in a state
where stack and expression have a type mismatch) or  reach an ill-formed stack.

Induction over machine state transitions proves the \textit{ok} part,
the productivity is proved by case analysis on the \textit{ok} rules and
the state well-formed-ness (checking all transition cases for the non-final case).

\subsection*{Correctness according to \PCF}

Any \PCF expression corresponds to a well-formed start state $\eval{\varepsilon}{e}$,
it is safe to evalute \PCF expressions. Does it do the right thing then?
(\textbf{correctness}, i.e. (\textbf{soundness} and \textbf{completeness}))

\begin{itemize}
\item[\bf Complete] Any \PCF expression that evaluates to a value does so with the K machine
\item[\bf Sound] Anything that the K machine evaluates to a final state is a \PCF expression that evaluates to the corresponding final value
\end{itemize}

For \textbf{completeness}, an induction over the evaluation in \PCF is required,
giving rise to
a) generalising to any stack instead of the empty one, and 
b) using an \textit{evaluation dynamics} of \PCF (i.e. values not evaluations)
to be able to apply the induction hypothesis.

We effectively prove (Lemma 28.2):
if $ e \Downarrow v $ then $ \eval{k}{e} \mapsto^* \stack{k}{v}$ for every stack $k$,
and assume the $\Downarrow$ evaluation dynamics relates this to $e \mapsto^* v$.

\bigskip
For \textbf{soundness}, a notion of "unravels-to" is introduced to abstract from multistep evaluation:
$s \looparrowright e$ will give us that if state $s$ is final then $e$ is a value,
and otherwise, a transition to $s' \looparrowright e'$ will lead to a value $v$
i.e. (Lemma 28.3): if $ s \mapsto s'$ and $s \looparrowright e, s' \looparrowright e'$
then $e \mapsto^* e'$.

This is argued using another, complicated, judgement $\bowtie$, which effectively 
describes the  stack evaluation and relates it to \PCF evaluation (Lemma 28.6).
\end{document}
