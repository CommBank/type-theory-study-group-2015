\documentclass[10pt,a4paper]{article}

\usepackage[utf8]{inputenc}
\usepackage[english]{babel}
\usepackage{amsmath}
\usepackage{amsfonts}
\usepackage{amssymb}
\usepackage{graphicx}
\usepackage[left=2cm,right=2cm,top=2cm,bottom=2cm]{geometry}

\usepackage[usenames,dvipsnames]{xcolor}
\usepackage{xspace}

\usepackage{proof}

\newcommand{\comment}[2]{{$\spadesuit${\bf #1: }{\sf #2}$\spadesuit$}}

\author{Jost Berthold}
\title{Overview of all PFPL languages}

\begin{document}

\section{Syntax}

\subsection{Language E of Expressions}

% some definitions for math typesetting
\newcommand{\ttt}[1]{\mbox{\tt #1}}
\newcommand{\tttt}[1]{\ \ttt{#1}\ }

\begin{tabular}{p{1.3cm}llp{6cm}}
\textbf{Lan-guage}& \textbf{Types} & \textbf{Expressions} & Comments\\
\hline
\textbf{E} (Ch.4)&
$\begin{array}{l}
  \ttt{num}\\
  \ttt{str}\\
\end{array} $ &
$\begin{array}{l}
x\\
\ttt{num}[n]\\
\ttt{str}[s]\\
e_1 + e_2\\
e_1 * e_2\\
e_1 \hat{\ }\ e_2\\
\ttt{let}\ x \tttt{be} e_1 \tttt{in} e_2
\end{array} $ &
 \\
\hline
\color{green}{
\textbf{ED}  (Ch.8.1)} &
&
\color{green}{
$\begin{array}{l}
\ttt{fun}\ f(x : \tau_1) : \tau_2 \ = \ e_1 \tttt{in} e_2\\
e_1(e_2)
\end{array} $
} &
\textcolor{green}{
Limited extension, superceded by next:
First-order functions, with their names
are from a different variable supply here.
}\\
\hline
\color{blue}{
\textbf{EF}  (Ch.8.2)} &
\textcolor{blue}{
$\begin{array}{l}
\tau_1 \rightarrow \tau_2
\end{array} $
} &
\color{blue}{
$\begin{array}{l}
\lambda (x:\tau) e\\
e_1(e_2)
\end{array} $
} &
\textcolor{blue}{
Full functions as first-class citizens, with variable names.
}\\
\hline
\end{tabular}

\subsection{Language T of G\"odel total functions}
\begin{tabular}{p{1.3cm}llp{6cm}}
\textbf{Lan-guage}& \textbf{Types} & \textbf{Expressions} & Comments\\
\hline
\textbf{T} (Ch.9)&
$\begin{array}{l}
  \ttt{nat}\\
  \tau_1 \rightarrow \tau_2\\
\end{array} $ &
$\begin{array}{l}
x, \ttt{z}, \ttt{s}(e)\\
\ttt{rec}\ x \ (\ttt{z} \hookrightarrow e_0, \ttt{s}(x) \tttt{with} y \hookrightarrow e_1)\\
\lambda (x:\tau) e\\
e_1(e_2)
\end{array} $ &
\textbf{T:} Total functions (limited recursion) \\
\hline
\color{gray}{
\textbf{Pairs} (Ch.10)} &
\color{gray}{
$\begin{array}{l}
  \ttt{unit}\\
  \tau_1 \times \tau_2\\
\end{array} $ } &
\color{gray}{
$\begin{array}{l}
()\\
<e_1, e_2>\\
e.\ttt{l}\\
e.\ttt{r}
\end{array} $ } &
\color{gray}{
Pairs, generalised in next extension
}\\
\hline
\color{blue}{
\textbf{Products} (Ch.10)} &
\color{blue}{
$\begin{array}{l}
  <\tau_i>_{i\in I}
\end{array} $ } &
\color{blue}{
$\begin{array}{l}
<e_i>_{i\in I}\\
e.i
\end{array} $ } &
\color{blue}{
$I$ a finite index set
}\\
\hline
\color{gray}{
\textbf{Alter-native} (Ch.11)} &
\color{gray}{
$\begin{array}{l}
  \ttt{void}\\
  \tau_1 + \tau_2\\
\end{array} $ } &
\color{gray}{
$\begin{array}{l}
\ttt{abort}\\
\ttt{l}.e\\
\ttt{r}.e\\
\ttt{case}\ e (\ttt{l}.x_1 \hookrightarrow e_1, \ttt{r}.x_2 \hookrightarrow e_2)
\end{array} $ } &
\color{gray}{
choice between two things, generalised in next extension
}\\
\hline
\color{purple}{
\textbf{Sum} (Ch.11)} &
\color{purple}{
$\begin{array}{l}
  <\tau_i>_{i\in I}
\end{array} $ } &
\color{purple}{
$\begin{array}{l}
\ttt{i}.e\\
\ttt{case}\ e\ <\ttt{\i}.x_i \hookrightarrow e_i>_{i\in I}\\
\end{array} $ } &
\color{purple}{
Choice from finite index set $I$.
Can express Booleans and Enums.
}\\
\hline
\color{violet}{
\textbf{Infi-nite} (Ch.14)} &
\color{violet}{
./.}
&
\color{violet}{
$\begin{array}{l}
\ttt{map}_{t.\tau}(x.e')\ e
\end{array} $ } &
\color{violet}{
The general type operation may use $+,\ \times$, $\ttt{unit}$ and $\ttt{void}$
from before. Restricted to \textit{positive} operation.
}\\
\hline
\end{tabular}

\subsection{Language family PCF of (general) recursive functions}

\begin{tabular}{p{1.3cm}llp{6cm}}
\textbf{Lan-guage}& \textbf{Types} & \textbf{Expressions} & Comments\\
\hline
\textbf{PCF} (Ch.19)&
$\begin{array}{l}
  \ttt{nat}\\
  \tau_1 \rightarrow \tau_2\\
\end{array} $ &
$\begin{array}{l}
x\\
\ttt{z}\\
\ttt{s}(e)\\
\ttt{ifz}\ e \ (e_0, \ x.e_1)\\
\lambda (x:\tau) e\\
e_1(e_2)\\
\ttt{fix}\ (x: \tau) \tttt{is} e\\
\end{array} $ &
 \\
 \hline
\color{blue}{
\textbf{FPC} (Ch.20)} &
\textcolor{blue}{
$\begin{array}{l}
t\\
\ttt{rec} t \tttt{is} \tau\\
\end{array} $
} &
\color{blue}{
$\begin{array}{l}
\ttt{fold}_{t.\tau}(e)\\
\ttt{unfold}(e)\\
\end{array} $
} &
\textcolor{blue}{
Full functions as first-class citizens, with variable names.
}\\
\hline

\end{tabular}

\newpage

\section{Typing}

\newcommand{\turnstile}{\vdash}

\begin{tabular}{p{1cm}cp{6cm}}
\textbf{Lan-guage}& \textbf{Rules} & Comments\\
\hline\\
\textbf{E} (Ch.4)&
$
\infer{\Gamma, x:\tau \turnstile x:\tau}{}
\qquad
\infer{\Gamma \turnstile \ttt{str}[s]: \ttt{str}}{}
\qquad
\infer{\Gamma \turnstile \ttt{num}[n]: \ttt{num}}{}
$ & Typing axiom and atoms
\\
&
$
\infer{\Gamma \turnstile \ttt{plus}(e_1,e_2):\ttt{num}}
	  {\Gamma\turnstile e_1:\ttt{num} & \Gamma\turnstile e_2:\ttt{num} }
\quad
\infer{\Gamma \turnstile \ttt{times}(e_1,e_2):\ttt{num}}
	  {\Gamma\turnstile e_1:\ttt{num} & \Gamma\turnstile e_2:\ttt{num} }
$ & \texttt{num} operations
\\[2ex]
&
$
\infer{\Gamma \turnstile \ttt{cat}(e_1,e_2):\ttt{str}}
	  {\Gamma\turnstile e_1:\ttt{str} & \Gamma\turnstile e_2:\ttt{str} }
\quad
\infer{\Gamma \turnstile \ttt{len}(e):\ttt{num}}
	  {\Gamma\turnstile e:\ttt{str} }
$ & conversions
\\[2ex]
&
$
\infer{\Gamma \turnstile \ttt{let}\ x \tttt{be} e_1 \tttt{in} e_2: \tau_2}
	  {\Gamma\turnstile e_1:\tau_1 & \Gamma, x:\tau_1 \turnstile e_2:\tau_2 }
$ & local binding
\\
\hline
\color{green}{
\textbf{ED}  (Ch.8.1)} &
\color{green}{
\ldots
} &
\textcolor{green}{
\comment{JB}{TODO}}\\
\hline
\end{tabular}

\end{document}